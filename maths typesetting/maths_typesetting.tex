\documentclass{article}
% importing amsmath package for using it later
\usepackage{amsmath}

\begin{document}
  first time written maths equations in latex
  using inline and environment type maths setting\\
  $$f(x)=x$$ %inline method
  \begin{equation}
    % without using any package hence the number will come automatically
    f(x) = x^2+2x+1
  \end{equation}
  \begin{equation*}
    % using amsmath package to remove the automatic numbering
    f(x) = x^2+2x+1
  \end{equation*}
  \begin{align*}
    % using align* to write multiple equations simultaneously with alignment on &
    a+b&=c\\
    b+d&=e
  \end{align*}
  \begin{align*}
    % another example to show alignment
    a+b&=c\\
    b&=d+e
  \end{align*}
  \begin{align*}
  c &= \begin{matrix}
    % matrix without braces
    1&2\\
    2&2
  \end{matrix}\\
  u &=[
  \begin{matrix}
    % using small braces not suitable
    1&2\\
    2&2
  \end{matrix}
  ]\\
  % now typesetting using commands
  v&=\left[
  \begin{matrix}
    1&2\\
    2&2
  \end{matrix}
  \right]\\
  G(x)&=\left(\frac{1}{x^3}\right)\\
  F(\alpha) &= \int^\alpha_1\frac{1}{x}dx
  \end{align*}
  % now applying all the different commands for different common functions
  \begin{align*}
    % firstly all trignometric functions
    f(x)&=\sin{x}\\
    % below command is different in the sense that sin is written in italics and not User friendly
    g(x)&=sinx\\  
    h(x)&=\csc{x}\\
    %now coming the integral type ones
    a &= \int_a^bxdx\\ %here lim a will be lower lim
    c &= \int^a_bxdx\\ %here a will be upper lim
    h(x) &= \int_Dxdx\\
    u(x,y) &= \iint_D f(x,y)dxdy\\
    v &= \oint_DFds\\
    %now some inbuilt matrix command
    M &= \begin{vmatrix}
      1&0\\
      0&1
    \end{vmatrix}\\
    N&=\begin{bmatrix}
      1&0\\
      0&1
    \end{bmatrix}\\
    D&=\begin{pmatrix}
      1&0\\
      0&1
    \end{pmatrix}\\
    c&=\det{\begin{vmatrix}
      1&0\\
      0&1
    \end{vmatrix}}\\
    % using some dot type commands
    P&=\begin{bmatrix}
      1&0&0&0\\
      1 & 0 & \cdots & 0\\
      \vdots&\vdots&\ddots&\vdots\\
      1&0&0&0
    \end{bmatrix}\\
    % using some miscellanous function
    f(x)&=\log{x}\\
    g(x)&=\log_a{x}\\
    h(x)&=\sqrt{x}\\
    u(x,n)&=\sqrt[n]{x}\\
    \phi(x)&=\frac{f(x)}{g(x)}
  \end{align*}
  
  Now above are some few commands for typesetting maths  
\end{document}
